\documentclass[journal,12pt,twocolumn]{IEEEtran}
\usepackage{amsmath,amssymb,amsfonts,amsthm}
\usepackage{txfonts}
\usepackage{tkz-euclide}
\usepackage{listings}
\usepackage{gvv}
\usepackage[latin1]{inputenc}
\usepackage{array}
\usepackage{pgf}
\usepackage{lmodern}

\begin{document}
\bibliographystyle{IEEEtran}

\title{DISCRETE 11.9.3 Q-4}
\author{EE23BTECH11066 - Yakkala Amarnath Karthik
}
\maketitle

\bibliographystyle{IEEEtran}
\textbf{Question:} \\ \\The $4^{th}$ term of a G.P. is square of its second term, and the first term is -3. Determine its $7^{th}$ term, and find the Z transform of the series.\\

\textbf{Solution:} \\ \\
Let, first term of this G.P.\brak{ X \brak0} be a.
\\Given, the first term is -3.\\ i.e. a= -3 (given).......(1) \\ 
Let r be the common ratio of G.P.\\
Given that the fourth term of G.P. is square of its second term.\\
\\We know that the general term of a G.P. can be written as : X(n)=$ar^{n}$......(2)
\begin{align}
 X\brak3=(X\brak1)^2\hspace{0.5cm}(Given)......(3) 
 \end{align}
 \hspace{3cm} substituting (2) in (3),
 \begin{align}
 ar^{3}=(ar^{1})^2\\
ar^3=a^2r^2\\
r=a\\
(from\brak1)\hspace{0.4cm}r=-3.......(4)\\
\text7^{th} term(X\brak6)=ar^{6}
\end{align}
\hspace{3cm} from (1) and (4)
\begin{align}
X\brak6=\brak{-3}\brak{-3}^6\\
X\brak6=\brak{-3}^7=-2187
\end{align}
\textbf{So $7^{th}$ term of the G.P. is -2187.}\\
\bigskip
\\ \\ \\ \\ \\ \\ \\ \\ \\ \\ 
  
Finding Z transform : 

\begin{align}
    X \brak{z} & = \sum_{n=-\infty}^{\infty} x \brak{n}   z^{-n} \\
    & = \sum_{n=-\infty}^{\infty} ar^n  u \brak{n}   z^{-n} \\
    & = \sum_{n=0}^{\infty} ar^n  z^{-n} \\
    & = a( 1 + rz^{-1} + r^{2}z^{-2} + r^{3}z^{-3} +...)\\
    & = \frac{a}{{1 - rz^{-1}}} 
\end{align}
\hspace{3cm}\cbrak{ROC:rz^{-1}<1} \\
So, the Z-transform of the given series is
$X(n)=\frac{a}{1-rz^{-1}}= \frac{-3}{1+3z^{-1}}$.\\
\begin{figure}[ht]
        \centering
        \includegraphics[width=0.45\textwidth]{graph.jpeg}
        \centering
        \caption{    Graph showing first 8 terms of the GP}
    \end{figure} 
    
\begin{table}[ht]
  \centering
  \begin{tabular}{|c|c|c|}
    \hline
    \textbf{Variable} & \textbf{Description} & \textbf{value}\\
    \hline
    a & first term of G.P. & -3 \\
    \hline
    r & Common ratio of G.P. & -3 \\
    \hline
    X(n) & general term of the G.P. & $ar^{n}$ \\
    \hline
  \end{tabular}
  \caption{A Table with input parameters}
  \label{tab:1}
\end{table}
\end{document}
